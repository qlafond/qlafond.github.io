\documentclass[parskip=full]{scrartcl}
\usepackage[inner=2cm,outer=2cm,top=2.5cm,bottom=2.5cm]{geometry}
\usepackage[utf8]{inputenc}
\usepackage{lmodern}
\usepackage{physics}
\usepackage{amsmath}
\usepackage{amssymb}
\usepackage{ulem}
\usepackage{mathrsfs}
\usepackage{amsthm}
\usepackage{siunitx}
\usepackage{tikz}
\usepackage{enumitem}
\usepackage{quiver}
\usepackage{amsthm}
\usepackage{mathtools}
\usepackage{bm}
\usepackage[section]{placeins}
\usepackage{graphicx}
\graphicspath{ {./image/} }
\DeclareMathSymbol{\sm}{\mathbin}{AMSa}{"39}
\usetikzlibrary{decorations.markings}
\makeatletter
\renewcommand*\env@matrix[1][*\c@MaxMatrixCols c]{%
  \hskip -\arraycolsep
  \let\@ifnextchar\new@ifnextchar
  \array{#1}}
\makeatother
\makeatletter
\renewcommand*\env@matrix[1][\arraystretch]{%
  \edef\arraystretch{#1}%
  \hskip -\arraycolsep
  \let\@ifnextchar\new@ifnextchar
  \array{*\c@MaxMatrixCols c}}
\makeatother
\newcommand{\myeq}{\mathrel{\overset{\makebox[0pt]{\mbox{\normalfont\tiny\sffamily def}}}{=}}}
\newcolumntype{C}{>{$}c<{$}}
\newcommand{\R}{\mathbb R}
\newcommand{\C}{\mathbb C}
\newcommand{\RP}{\mathbb R\mathbb P}
\newcommand{\Mat}{\text{Mat}}
\newcommand{\Aut}{\text{Aut}}
\newcommand{\dvolg}{\text{dvol}_g}
\renewcommand{\L}{\mathscr{L}}
\newcommand{\Id}{\text{Id}}
\newcommand{\N}{\mathbb{N}}
\newcommand{\Hom}{\text{Hom}}
\newcommand{\Z}{\mathbb{Z}}
\newcommand{\K}{\mathbb{K}}
\newcommand{\Oplus}{\bigoplus}
\newcommand{\Otimes}{\bigotimes}
\newcommand{\End}{\text{End}}
\newcommand{\sbullet}[1][.5]{\mathbin{\vcenter{\hbox{\scalebox{#1}{$\bullet$}}}}}
\newlist{problems}{enumerate}{1}
\setlist[problems,1]{label=\roman*), leftmargin=2em, itemsep=1.5em}

\newtheorem{theorem}{Theorem}[section]
\newtheorem{corollary}{Corollary}[section]
\newtheorem{proposition}{Proposition}[section]
\newtheorem{lemma}{Lemma}[section]
\numberwithin{equation}{section}
\theoremstyle{definition}
\newtheorem{example}{Example}[section]
\newtheorem{definition}{Definition}[section]
\providecommand*\definitionautorefname{Definition}
\providecommand*\lemmaautorefname{Lemma}
\providecommand*\propositionautorefname{Proposition}
\providecommand*\exampleautorefname{Example}

\title{Calculus I}
\subtitle{Challenge Homework Set I}
\author{}
\date{\today}
\setkomafont{subtitle}{\normalfont\Large}
\setkomafont{title}{\normalfont\large}
\setkomafont{section}{\normalfont\large}
\setkomafont{subsection}{\normalfont\large}
\setkomafont{subsubsection}{\normalfont\large}

\begin{document}
\maketitle
Provide \textbf{handwritten} answers on a separate sheet of paper. Typed answers will not be accepted.
For full credit correct answers should be clear, legible, include explanations for your reasoning, and 
show all relevant work. You are allowed to make use of outside resources, including the internet, and
friends, but you must cite your sources.
\textbf{Textbook Problems}:

Ch 4: 113-128, 316,  322, 338-340
\begin{problems}
  \item Find the critical points of the following functions, and evaluate whether the 
  critical points are maxima or minima. 
  \begin{itemize}
    \item [$a)$]$\cos (2x)$
    \item [$b)$] $\sin (5x)$
    \item [$c)$] $\sin (|x|)$
    \item [$d)$] $|\cos x|$
  \end{itemize}
  \item In this problem we consider optimizing the volume or surface area of certain shapes. \textbf{Hint:} Draw pictures!
  \begin{itemize}
    \item [$a)$] Find the largest volume of a cylinder that fits into a cone of radius $r$ 
    and height $h$.
    \item [$b)$] Find the dimensions of a cylinder with volume $16\pi\operatorname{m}^2$ that has the largest surface area.
    \item [$c)$] Find the dimensions of a right cone with surface area $4\pi\operatorname{m}^2$ that has the largest volume.
    \item [$d)$] Suppose that total surface area of a cube and and sphere is $1\operatorname{m}^3$. Find the dimensions 
    of the cube and sphere such that the total volume is maximized.
  \end{itemize}
  \item For this problem recall that $(x_1,y_1)$ and $(x_2,y_2)$ are two points in the plane, then the 
  distance between them is given by:
  \begin{align*}
    d=\sqrt{(x_1-x_2)^2+(y_1-y_2)^2}
  \end{align*}
  Using this, answer the following questions:
  \begin{itemize}
    \item Where is the line $y=5-2x$ closest to the origin?
    \item Where is the parabola $y=x^2$ closest to the point $(2,0)$?
    \item Where is the cubic $y=x^3$ closest to the point $(2,2)$?
  \end{itemize}
  \item An object with mass $m$ is dragged along a horizontal
  plane by a force acting along a rope attached to the object.
  If the rope makes an angle $\theta$ with a plane, then the magnitude of the force is:
  \begin{align*}
    F=\frac{\mu m g}{\mu \sin \theta+\cos\theta}
  \end{align*}
  where $g$ is the acceleration due to gravity, and $\mu$ is a dimensionless 
  constant called the coefficient of friction. For what value of $\theta$ 
  is $F$ minimized?
\end{problems}
\end{document}


